\documentclass[a4paper,11pt]{article}

\usepackage{graphicx}
\usepackage{amsmath,amssymb,amsthm}
\usepackage[round]{natbib}
\usepackage{url}
\usepackage{xspace}
\usepackage[left=25mm,top=23mm, right=25mm]{geometry}
\usepackage{algorithmic}
\usepackage{subcaption}
\usepackage{mathpazo}
\usepackage{booktabs}
\usepackage{hyperref}
\usepackage{caption}
\usepackage{subcaption}

\title{Doma\v ca naloga iz matemati\v cne fizike}
\author{Maks Kamnikar}
% \date{\today}

\begin{document}
    \maketitle

    % Uvod: Naloga
    \textbf{Naloga:}
    \newline
    \v Stiri enako dolge ravne prevodne nano-\v zi\v cke staknemo v kvadraten okvir in nanj
    nanesemo nekaj elektronov.
    \newline
    Kako se razporedijo?
    \newline
    Kak\v sno je elektri\v cno polje okvira?
    \newline
    Ali limitira s \v stevilom elektronov k stalni obliki?


    \vspace{30pt}
    % Uvod
    Moja naloga je poiskati optimalno razporeditev elektronov na kvadratnem okvirju.
    To je tak\v sna razporeditev, pri kateri je skupna potencialna energija sistema elektronov
    najmanj\v sa.

    \vspace{30pt}
    Za potencialno energijo i-tega elektrona v polju j-tega elektrona velja sorazmernost
    $$ W_{i,j} \propto \frac{1}{\left\vert\mathbf{r}_{i} - \mathbf{r}_{j}\right\vert}. $$

    \noindent
    Torej je potencialna energija i-tega elektrona v polju vseh ostalih elektronov
    $$ W_{i} = \sum_{j \neq i} W_{i,j}. $$

    \noindent
    I\v s\v cem razporeditev elektronov pri kateri bo vsota vseh njihovih potencialnih energij
    $$ W = \sum_{i} W_{i} \propto \sum_{i} \sum_{j \neq i} \frac{1}{\left\vert\mathbf{r}_{i} - \mathbf{r}_{j}\right\vert} $$
    najmanj\v sa.

    \vspace{10pt}
    Elektron na kvadratnem nosilcu je v lokalnem minimumu potenciala, ko nanj deluje sila,
    ki ima neni\v celno kve\v cjemu komponento pravokotno na nosilec, in elektrona ne potiska
    vzdol\v z nosilca.
    Sila j-tega elektrona na i-ti elektron vzdol\v z stranice na kateri le\v zi i-ti elektron je
    $$ F_{i,j} =\mathbf{F}_{i,j} \cdot \hat{\mathbf{a}}_i= \frac{e_{0} ^ 2}{4 \pi \varepsilon_{0} \left\vert \mathbf{r}_{i} -
    \mathbf{r}_{j} \right\vert ^2}
    \frac{\mathbf{r}_{i} - \mathbf{r}_{j}}{\left\vert \mathbf{r}_{i} - \mathbf{r}_{j} \right\vert} \cdot \hat{\mathbf{a}}_i
    \propto \frac{\cos \varphi_{i,j}}{\left\vert \mathbf{r}_{i} -
    \mathbf{r}_{j} \right\vert^3} , $$

    \noindent
    kjer je $\hat{\mathbf{a}}_i$ enotski vektor vzporeden s stranico na kateri le\v zi i-ti elektron, $\varphi_{i,j}$ pa kot med
    vektorjema $ \mathbf{r}_{i} - \mathbf{r}_{j}$ in $\hat{\mathbf{a}}_i$. Oziroma

    $$ F_{i, j} \propto \frac{\Delta y}{(\Delta x_{i,j} ^2 + \Delta y_{i,j} ^2)^2} , $$

    \noindent
    kjer je $\Delta x_{i,j} ^2 + \Delta y_{i,j} ^2 = \left\vert \mathbf{r}_{i} - \mathbf{r}_{j} \right\vert ^2 = (x_i - x_j)^2
    + (y_i - y_j) ^2$, $\cos{\varphi_{i,j}} = \frac{\Delta y}{\sqrt{\Delta x_{i,j} ^2 + \Delta y_{i,j} ^2}}$.


    \noindent
    Iskal sem tako razporeditev elektronov po okvirju, da velja neenakost
    $$F_i = \sum_{j \neq i} F_{i,j} \leq \varepsilon $$

    \noindent
    za vsak $i = 1, 2, ..., n$, kjer je $n$ \v stevilo elektronov na okvirju in $\varepsilon > 0, \varepsilon \ll 1$.


    % Jedro
    \vspace{10pt}
    To razporeditev sem iskal tako, da sem simuliral $n$ elektronov na kvadratnem okvirju, pri \v cemer
    sem premikal elektrone vzdol\v z stranic okvirja v smeri rezultante sil nanje. Simulacijo sem kon\v cal, ko
    so bile sile na elektrone s komponento vzdol\v z stranic dovolj majhne za \v zeleno numeri\v cno natan\v cnost
    ($ F_i \leq \varepsilon $),
    oz. ko sta si bili stanji dveh zaporednih korakov simulacije dovolj podobni.

    \vspace{10pt}
    Da bi s simulacijo na\v sel le stabilna ravnovesna stanja, sem sistem, ko je pristal v morebitnem labilnem stanju
    izmaknil iz njega in preveril, da se ravnovesje spet vzpostavi v istih to\v ckah. To sem storil tako,
    da sem vsakemu delcu, ki ni le\v zal v ogli\v s\v cu kvadratnega nosilca, naklju\v cno spremenil
    pozicijo vzdol\v z stranice na kateri je le\v zal. Psevdonaklju\v cna Pythonova knji\v znica \texttt{random}, ki sem
    jo uporabljal, je dovolj naklju\v cna za ta namen, saj je verjetnost, da bi sistem s to perturbacijo ponesre\v ci
    premaknil ravno v neko drugo labilno stanje, zanemarljiva. Stabilno ravnovesno stanje ni eno, ampak zaradi simetrije
    obstaja 8 stabilnih, energetsko enakovrednih, optimalnih stanj. 4 zaradi zrcaljenj in 4 zaradi rotacij.

    % \begin{center}
    %     % \includegraphics[scale=0.5]{{Figure_1.pdf}}
    %     \includegraphics[width=\textwidth]{{Figure_1.pdf}}
    % \end{center}

    \begin{figure}[h!]
        % \includegraphics[scale=0.45]{{Figure_1.pdf}}
        \includegraphics[width=\textwidth]{{Figure_1.pdf}}
        \begin{center}
            \textbf{Slika 1}: Primeri razli\v cnih stabilnih in labilnih ravnovesij.
        \end{center}
    \end{figure}

    Slika 1.1 in 1.2 prikazujeta dve distribuciji dvajsetih elektronov, ki sta obe stabilni, kljub temu pa obe nimata iste
    potencialne energije. Najni\v zje energije imajo sistemi z delci v ogli\v s\v cih kvadratnega nosilca, saj
    imajo tako ve\v cje medsebojne razdalje. Slika 1.3 prikazuje stabilno ravnovesno stanje za \v stiri delce, Slika 1.4
    pa labilno stanje sistema \v stirih elektronov.

    \vspace{10pt}
    Zanimala so me stanja z minimalno energijo, se pravi tista, ki imajo \v cimve\v c delcev v ogli\v s\v cih kvadrata.
    Da sem zagotovil, da se bo sistem res zna\v sel v takem stanju, sem elektrone pred za\v cetkom simulacije nanesel
    na kvadratni okvir tako, da sem vsaj en elektron vedno postavil v eno od ogli\v s\v c; razporedil sem jih enakomerno
    po polarnem kotu za\v cen\v si s kotom $\frac{\pi}{4}$. Polarni kot i-tega elektrona $\theta_i$ v sistemu $n$
    elektronov je torej

    $$ \theta_i = \frac{2 \pi}{n} (i - 1) + \frac{\pi}{4} .$$

    \vspace{10pt}
    Zdaj so elektroni na\v sli svoje ravnovesne polo\v zaje, mi pa se spra\v sujemo, \v ce je kaj zanimivega v elektri\v cnem polju
    okoli teh elektronov. Vemo da je sila na i-ti elektron $\mathbf{F}_i = \mathbf{E} e_0$, prav tako pa zaradi ravnovesnega
    pogoja dr\v zi $F_i \leq \varepsilon \ll 1$. Torej je komponenta sile na i-ti elektron vzdol\v z stranice na kateri le\v zi
    zanemarljiva. Posledica tega je, da je neni\v celna lahko kve\v cjemu komponenta sile pravokotna na stranico kvadrata.
    Torej so silnice elektri\v cnega polja v to\v cki vsakega elektrona usmerjene ven iz kvadrata, normalno na njegove stranice.

    \begin{figure}[h!]
        % \includegraphics[scale=0.45]{{Figure_1.pdf}}
        \includegraphics[width=\textwidth]{{Figure_2-cc.pdf}}
        \begin{center}
            \textbf{Slika 2}: Silnice elektri\v cnega polja sistema stotih elektronov in njegove ekvipotencialne ploskve.
        \end{center}
    \end{figure}

    Na Sliki 2.1 vidimo, da nas teorija ni pustila na cedilu in, da so silnice polja res normalne na stranice kvadrata.
    Silnice prebadajo ekvipotencialne ploskve pravokotno ($\mathbf{E} = -\nabla V $), zato lahko iz Slike 2.2 potrdimo kar smo ugotovili iz Slike 2.1.
    Seveda so silnice normalne na stranico kvadrata le v tistih to\v ckah, kjer le\v zi elektron, v limiti zvezno
    porazporejenega naboja pa to velja za vsako to\v cko na kvadratnem okvirju.

    \vspace{10pt}
    Trditev, da silnice elektri\v cnega polja prebadajo stranice oz. izvirajo iz stranic kvadratnega okvirja normalno nanj, velja le
    v limiti neskon\v cno elektronov oziroma kadar je po prevodniku naboj razporejen zvezno. Za veliko \v stevilo elektronov na
    okvirju \v se kar dobro velja, da so silnice pravokotne na okvir, za majhno \v stevilo delcev pa ta trditev obupno odpove.

    \begin{center}
    \includegraphics[width=\textwidth]{{Figure_3-cc.pdf}}
    \textbf{Slika 3}: Silnice elektri\v cnega polja sistema petih elektronov in njegove ekvipotencialne ploskve.
    \end{center}

    Na Sliki 3.1 jasno vidimo, da silnice elektri\v cnega polja ne prebadajo stranic kvadrata pod pravim kotom. Iz Slike 3.2
    pa opazimo, da ekvipotencialne ploskve niso niti pribli\v zno vzporedne stranicam kvadrata.

    \begin{center}
    \includegraphics[width=\textwidth]{{Figure_4-cc.pdf}}
    \textbf{Slika 4}: Silnice elektri\v cnih polj razli\v cno \v stevil\v cnih sistemov.
    \end{center}

    Na Sliki 4 vidimo, da elektri\v cno polje okvira res limitira k neki funkciji z nara\v s\v cajo\v cim \v stevilom elektronov,
    saj sta Sliki 4.3 in 4.4 skoraj enaki.
    \v Ce imamo v sistemu le 20 elektronov, kot jih imamo v Sliki 4.1, je o\v citno, da elektri\v cno polje ni povsod  pravokotno
    na kvadratni okvir, pri sistemu z 80 elektroni pa to \v ze kar dobro dr\v zi.

    \vspace{10pt}
    Ena on bistvenih pomankljivosti grafov, kjer na okvir gledamo v tlorisu, je ta, da ne vidimo kaj se dogaja s silnicami
    v smeri normalni na ploskev kvadrata. Na prvi pogled izgleda kot, da bi bil v sredi\v s\v cu kvadrata pozitiven naboj
    iz katerega izvirajo silnice, ki so usmerjene proti stranicam. Seveda temu ni tako. Te silnice izvirajo iz neskon\v cnosti
    nad in pod ravnino kvadrata, vendar zaradi projekcije na dvodimenzionalen graf izgleda kot, da se v sredi\v s\v cu kvadrata
    skriva nek pozitiven naboj.

    \begin{center}
    \includegraphics[width=1\linewidth]{Figure_5.pdf}
    \textbf{Slika 5}: Silnice elektri\v cnega polja sistema stotih elektronov.
    \end{center}

    Na Sliki 5 vidimo, kak\v sno je elektri\v cno polje okvira nad in pod njim. O\v citno je, da v sredi\v s\v cu kvadrata
    ni nobenih nabitih delcev. Seveda se polje \v siri nesko\v cno v vse smeri, vendar je na sliki zaradi preglednosti narisan le
    ozek centralni del.


\end{document}
